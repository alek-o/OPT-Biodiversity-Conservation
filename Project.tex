\documentclass{article}
\usepackage{amsmath} % Required for math environments
\usepackage{amssymb} % Provides \mathbb and additional math symbols

\begin{document}
%=====================================================================
% Sets
%=====================================================================

\subsection*{Sets}

\begin{itemize}
    \item $I$ : set of grid cells, indexed by $i$.
    \item $S = \{A, M, E, O\}$ : set of species.
    \item $\mathrm{Adj} \subseteq I \times I$ : adjacency relation; 
          $(i,j) \in \mathrm{Adj}$ if and only if cells $i$ and $j$ 
          share a side (4-neighborhood).
    \item $K_s$ : set of original colonies (territories) of sp.56ecies $s$.
          Each $k \in K_s$ represents one connected component 
          of originally occupied cells.
\end{itemize}


%=====================================================================
% Parameters
%=====================================================================

\subsection*{Parameters}

\begin{itemize}
    \item $area_i \in \mathbb{R}_{+}$ : area of cell $i$. $area_i\geq 0.56$ is maximum.
    \item $orig\_has_{i,s} \in \{0,1\}$ : 1 if species $s$ originally occurs in cell $i$.

    \item $cost^{T}_{i,s} \in \mathbb{R}_{+}$ : adaptation (territory expansion) cost 
          for species $s$ to occupy cell $i$.
    \item $cost^{C}_{i} \in \mathbb{R}_{+}$ : corridor creation cost in cell $i$.

    \item $minArea_s \in \mathbb{R}_{+}$ : minimum viable total territory area for species $s$.
    \item $meanArea_s \in \mathbb{R}_{+}$ : “satisfactory” target area for species $s$.
    \item $maxArea_s \in \mathbb{R}_{+}$ : maximum allowed territory area for species $s$.

    % Colony-level area requirements:
    \item $minArea_{k,s} \in \mathbb{R}_{+}$ : minimum required area for colony $k \in K_s$.
    \item $meanArea_{k,s} \in \mathbb{R}_{+}$ : satisfactory target area for colony $k \in K_s$.
    \item $maxArea_{k,s} \in \mathbb{R}_{+}$ : maximum allowed area for colony $k \in K_s$.

    \item $conflict(s,s') \in \{0,1\}$ : equals $1$ if species $s$ and $s'$ 
          cannot overlap (e.g.\ $M$ with $E$ or $O$), otherwise $0$.
    \item $suit_{i,s} \in [0,1]$ : suitability score of cell $i$ for species $s$ (1 = ideal habitat, 0 = unsuitable).
    \item $priority_s \in \mathbb{R}_{+}$ : priority coefficient of species $s$ (higher means greater conservation priority).
    \item $L_{\max} \in \mathbb{Z}_{+}$ : maximum allowed corridor length (in cells).
\end{itemize}


%=====================================================================
% Decision Variables
%=====================================================================

\subsection*{Decision Variables}

\begin{itemize}

    % TERRITORY
    \item $T_{i,s} \in \{0,1\}$ : 
          1 if cell $i$ belongs to the \emph{final} territory of species $s$ 
          (original or expanded), 0 otherwise.

    \item $E_{i,s} \in \{0,1\}$ : 
          1 if species $s$ newly expands into cell $i$ 
          (i.e.\ $i$ was not originally occupied), 0 otherwise.

    % CORRIDOR
    \item $C_i \in \{0,1\}$ : 
          1 if a corridor is built in cell $i$, 0 otherwise.

    % CONNECTIVITY LINKING VARIABLES
    \item $\lambda_{(i,j),s} \in \{0,1\}$ :
          1 if species $s$ uses adjacency $(i,j) \in \mathrm{Adj}$ as a 
          corridor connection between two territories.
      
    \item $Reach_{i,s,k} \in \{0,1\} 
    \quad\text{for } i\in I,\; s\in S,\; k\in K_s,$:
          1 if colony $k$ of species $s$ can reach cell $i$ via corridors, 
          0 otherwise.

    \item $g_{i,j,s,k} \ge 0 $:
          corridor flow on arc $i\to j$for colony $k$ of species $s$.

    \item $Inject_{s,i,k} \ge 0$:
          flow injected at corridor cell $i$ for colony $k$ of species $s$.

    \item $use_{i,s}\in\{0,1\}$:
         species-level corridor-use; optional.

    \item $dist_{i,s,k} \in \{0,1,\dots,L_{\max}\}
\qquad\forall i\in I,\; s\in S,\; k\in K_s.$:
          distance (in cells) from colony $k$ of species $s$ to cell $i$ via corridors.

\end{itemize}


%=====================================================================
% Auxiliary Variables (derived, structural)
%=====================================================================

\subsection*{Auxiliary Variables}

\begin{itemize}
    \item $Area_{s} = \sum_{i \in I} area_i \, T_{i,s}$ :
          total final area of species $s$.

    \item $ExpArea_s = \sum_{i \in I} area_i \, E_{i,s}$ :
          total expanded area for species $s$.

    \item $Area_{k,s} = \sum_{i \in I} area_i \, T_{i,s,k}$ :
          area of colony $k$ of species $s$, where
          $T_{i,s,k}$ indicates membership of $i$ in colony $k$ 
          (colony-assignment variables introduced in the model section).

    \item $y_s = \sum_{(i,j)\in \mathrm{Adj}} \lambda_{(i,j),s}$ :
          number of corridor links used for species $s$.

    \item $n_s = |K_s|$ :
          number of original colonies of species $s$.

\end{itemize}


%=====================================================================
% Stage-related quantities
%=====================================================================

\subsection*{Stage-Related Quantities}

\begin{itemize}
    \item $\text{MinFeasible}_s = minArea_s$ : 
          minimum feasible area requirement for species $s$ (Stage~1).

    \item $\text{MeanTarget}_s = meanArea_s - minArea_s$ : 
          area above minimum toward satisfactory target (Stage~2).

    \item $\text{MaxPotential}_s = maxArea_s - minArea_s$ :
          maximum possible expansion above minimum (Stage~3).

    \item $ConnTarget_s = n_s - 1$ :
          maximum number of meaningful inter-colony connections for species $s$ 
          (tree structure).

\end{itemize}

\section{Objective function}

\subsection{Stage 1: Minimum viable territory}

% ============================
% Stage 1: Min-cost to satisfy minimal territory
% ============================

% Objective: minimize total expansion + corridor cost
% -------------------------
% Precomputed ratio parameter
% -------------------------
\subsection*{Precomputed parameter}
\begin{itemize}
  \item \(\rho_{i,s} := \dfrac{area_i}{cost^{T}_{i,s}}\) (precompute for each 0 and species; assume \(cost^{T}_{i,s}>0\)). 
        \textit{This is a constant parameter used to linearize area-per-cost preference.}
\end{itemize}

% -------------------------
% Stage 1: Objective (LaTeX)
% -------------------------
\subsection*{Stage 1 objective (minimal territory expansion, no priority)}

Stage~1 requires that each species attains its minimal territory.  
To achieve the required minima with best cost-effectiveness and habitat quality, we maximize the sum of per-cell benefit coefficients for newly expanded cells:

\[
\begin{aligned}
\text{(Stage 1)}\quad &\max \; 
    Z^{(1)} \;=\; \sum_{s \in S} \sum_{i \in I} \; 
      \underbrace{suit_{i,s}}_{\text{suitability}} \;
      \underbrace{\rho_{i,s}}_{\!\dfrac{area_i}{cost^{T}_{i,s}}\!} \;
      E_{i,s}
\end{aligned}
\]

subject to Stage~1 feasibility constraints, e.g.:

\[
\begin{aligned}
& \text{(a) Minimal area per species:} \\
& \sum_{i\in I} area_i \, E_{i,s} \;\le\; R_{\min,s} + A_{\max}^{\text{cell}}
      \quad \forall s \in S, \\[4pt]
& \text{where \(A_{\max}^{\text{cell}} := \max_{i\in I} area_i\).}
& \text{(b) Expansion feasibility (adjacency \& connectivity):} \\
& \text{(constraints ensuring each expanded cell is adjacent / reachable to its original patch)} \\[4pt]
& \text{(c) No-merge and conflict rules:} \\
& \text{(constraints preventing merges between original patches of same species and forbidding} \\
& \text{expansion that violates conflicts)} \\[4pt]
& \text{(d) Binary domains:} \\
& E_{i,s} \in \{0,1\},\; T_{i,s} \in \{0,1\},\; C_i \in \{0,1\},\; \ldots
\end{aligned}
\]

\[
\text{(Note)}\quad \rho_{i,s} \text{ and } suit_{i,s} \text{ are model parameters; the objective is linear in } E_{i,s}.
\]


\subsection{Stage 2 objective (satisfactory territory expansion with proportional corridor creation)}

% -------------------------
% Stage 2: Objective & auxiliary definitions
% -------------------------
\subsection*{Stage 2 objective (balanced territory \& corridors)}

% --- Auxiliary splits and deviation variables ---
\paragraph{Auxiliary definitions (per species $s$).}
\[
\begin{aligned}
& A_s \;=\; \sum_{i\in I} area_i \, T_{i,s} 
        \quad\text{(total final area of species } s\text{),}\\[4pt]
& A_{1,s} + A_{2,s} \;=\; A_s, 
        \quad 0 \le A_{1,s} \le n_s \cdot meanArea_s, \quad A_{2,s} \ge 0,\\[4pt]
& TB_s \;\le\; A_{1,s} - m_s, \qquad TB_s \ge 0, \qquad TB_s \le Rmean_s
        \quad\text{(territory between-min-and-mean)},\\[4pt]
& \text{choose } Y_s \text{ as either: }
    \begin{cases}
      \displaystyle Y_{cells,s} \;=\; \sum_{i\in I} use_{i,s}
              &\text{(corridor cells assigned to species } s\text{ - easier}),\\[6pt]
      \displaystyle Y_{edges,s} \;=\; \sum_{(i,j)\in \mathrm{Adj}} \lambda_{(i,j),s}
              &\text{(corridor edges used by species } s \text{ - preferred}).\\
    \end{cases}
\end{aligned}
\]

\paragraph{Soft proportional constraint (per species $s$).}
(If $n_s \le 1$ or $Rmean_s\le 0$, omit this constraint for that $s$ and set deviations to zero.)

\[
Y_s \cdot Rmean_s \;-\; TB_s \cdot (n_s - 1) \;=\; dev_{s}^{+} \;-\; dev_{s}^{-},
\qquad dev_{s}^{+},dev_{s}^{-} \ge 0.
\]

\paragraph{Connectivity certificate variables (recall / definitions).}
\[
ConnectedCount_s \;=\; \sum_{k\in K_s} conn_{s,k},
\]
where \(conn_{s,k}\in\{0,1\}\) indicates colony \(k\) of species \(s\) is certified connected
(via corridor flow constraints introduced in the model).

\paragraph{Connectivity reward scaling parameter. (An example choice.)}
\[
\gamma_{conn}=\frac{n_{s}\cdot meanArea_{s}-minArea_{s}}{n_{s}-1}
\]

% --- Stage 2 objective ---
\paragraph{Stage 2 objective (maximize).}
\[
\begin{aligned}
\text{(Stage 2)}\quad & \max \; Z^{(2)} \;=\;
  \underbrace{\frac{\sum_{s\in S} \Big( A_{1,s} \;+\; 0.7 \, A_{2,s} \Big)}{0.56}}_{\text{territory range-reward (full cell units)}}
  \;+\; \underbrace{\gamma_{\mathrm{conn}} \sum_{s\in S} ConnectedCount_s}_{\text{connectivity reward}} \\
&\qquad\qquad\qquad
  - \; \underbrace{\gamma_{\mathrm{cost}} \Big( \sum_{s\in S}\sum_{i\in I} cost^T_{i,s}\, E_{i,s}
         \;+\; \sum_{i\in I} cost^C_{i}\, C_i \Big)}_{\text{adaptation \& corridor costs}} \\
&\qquad\qquad\qquad
  - \; \underbrace{\gamma_{\mathrm{edge}} \sum_{(i,j)\in\mathrm{Adj}} \sum_{s\in S} \lambda_{(i,j),s}}_{\text{edge penalty (favor tree-like)}}\\
&\qquad\qquad\qquad
  - \; \underbrace{\gamma_{\mathrm{prop}} \sum_{s\in S} \big( dev_{s}^{+} + dev_{s}^{-} \big)}_{\text{penalize proportion deviation}}.
\end{aligned}
\]

% -------------------------
% Implementation / practical notes
% -------------------------
\subsection*{Implementation notes and recommended parameter choices}

\begin{itemize}
  \item \textbf{Choice of \(Y_s\):} for true tree semantics select \(Y_s := Y_{edges,s}\). If corridor costs and practical construction are cell-based, the simpler option is \(Y_s := Y_{cells,s}\). Use edges for proportionality to \(n_s-1\).
  \item \(\gamma_{\mathrm{conn}}\) should reflect how valuable connecting one colony is relative to adding area. As a baseline, it can be set to \(\gamma_{\mathrm{conn}} \approx\) average per-patch area-equivalent (e.g. mean cell-area).
  \item \(\gamma_{\mathrm{cost}}\) scales monetary costs into the same units as territory reward. If costs are already monetary and territory reward is in area units, normalize by converting one to the other or choose \(\gamma_{\mathrm{cost}}\) to produce desired trade-offs. Suggested equation: \(\gamma_{\text{costs}} = \frac{1}{|I|} \sum_{i \in I} \rho_{i,s} = \frac{1}{|I|} \sum_{i \in I} \frac{\text{cost}_{i,s}}{cost^T_{i,s}}
\).
  \item \(\gamma_{\mathrm{edge}}\) should be small (e.g. 0.01–0.1 in normalized units) to prefer trees but not overrule connectivity gains.
  \item \(\gamma_{\mathrm{prop}}\) controls how tightly the proportion is enforced: large values \(\Rightarrow\) near-equality, small values \(\Rightarrow\) loose. Start with a moderate value and tune.
  \item For species with \(n_s\le 1\) or \(Rmean_s\le 0\), set \(dev_{s}^{\pm}=0\) (no proportional constraint) and either fix \(Y_s=0\) (no corridors) or treat corridor reward as zero for that species.
  \item Maintain Stage~1 hard constraints (minimal territory) when solving Stage~2 (do not relax them).
\end{itemize}

% -------------------------
% Stage 3: Objective (fill up to max if budget remains)
% -------------------------
\subsection*{Stage 3 objective (top-up expansion to maxima)}

\paragraph{Auxiliary parameters (for Stage 3).}
\begin{itemize}
  \item $Budget_{\mathrm{rem}} \in \mathbb{R}_{+}$ : remaining monetary budget available for Stage~3 expansions.
  \item $\displaystyle Rmax_s \;:=\; n_s \, k_s \;-\; m_s$ : maximum remaining area (in cell-units or area-units) for species $s$ above the minimum.
  \item (Optional) $\; \delta \in (0,1]$ : diminishing-return coefficient for expansion beyond the "mean" level (use $\delta<1$ to downweight overly large expansions). If not used, set $\delta=1$.
\end{itemize}

\paragraph{Stage 3 objective (maximize additional expansion benefit).}
\[
\begin{aligned}
\text{(Stage 3)}\quad &\max \; Z^{(3)} \;=\;
    \sum_{s\in S}\; \sum_{i\in I} 
      \underbrace{suit_{i,s}}_{\text{suitability}}
      \underbrace{\rho_{i,s}}_{\,\dfrac{area_i}{cost^T_{i,s}}\,} \;
      E_{i,s}^{(3)}
\end{aligned}
\]
where $E_{i,s}^{(3)}$ are the Stage~3 decision binaries that equal 1 if cell $i$ is chosen for additional expansion for species $s$ during Stage~3 (cells already committed in Stages~1–2 should be fixed and excluded from $E_{i,s}^{(3)}$).

\paragraph{Subject to (core) Stage 3 constraints.}
\[
\begin{aligned}
&\text{(a) Per-species maximum cap:} \\
& \sum_{i\in I} area_i \; E_{i,s}^{(3)} \; \le \; Rmax_s
    \quad \forall s\in S, \\[6pt]
&\text{(b) Budget constraint:} \\
& \sum_{s\in S}\sum_{i\in I} cost^T_{i,s}\; E_{i,s}^{(3)} \;+\;
  \sum_{i\in I} cost^C_{i}\; C_i^{(3)} \; \le \; Budget_{\mathrm{rem}}, \\[6pt]
&\text{(c) Feasibility (adjacency / no-merge / conflict / corridor rules):}\\
&\text{(Stage-3 expansions must obey the same adjacency, connectivity and conflict constraints)} \\[6pt]
&\text{(d) Binary domain:} \\
&E_{i,s}^{(3)} \in \{0,1\}, \; C_i^{(3)} \in \{0,1\}, \; \dots
\end{aligned}
\]

\paragraph{Optional: diminish returns beyond mean.}
If you wish to reduce marginal benefit for expansions beyond the satisfactory (mean) level, split each species' additional expansion into two parts:
\[
\begin{aligned}
& A_{1,s}^{(\mathrm{tot})} \;=\; \sum_{i\in I} area_i \, T_{i,s} \quad\text{(current total after Stages 1–2)}\\[4pt]
& \text{Let } Exceed_{s} \;=\; \max\{0,\; A_{1,s}^{(\mathrm{tot})} - n_s\cdot meanArea_s \}
\end{aligned}
\]
and then weight expansions beyond the mean by $\delta$ in the objective:

\[
Z^{(3)} \;=\; \sum_{s}\sum_{i} suit_{i,s}\,\rho_{i,s}\, E_{i,s}^{(3, \, \le mean)} \;+\; 
           \delta \sum_{s}\sum_{i} suit_{i,s}\,\rho_{i,s}\, E_{i,s}^{(3, \, > mean)}
\]

(Implementation note: implement the split $E^{(3,\le mean)}$ and $E^{(3,> mean)}$ by comparing cumulative area to $n_s\cdot meanArea_s$ using standard linearization tricks.)

\paragraph{Notes.}
\begin{itemize}
  \item In a sequential workflow, variables/decisions already fixed by Stage~1 and Stage~2 should be fixed (constants) during Stage~3; the Stage~3 binary $E_{i,s}^{(3)}$ should only be available for cells not already selected.
  \item If corridor additions are permitted in Stage~3, include $C_i^{(3)}$ and associated corridor feasibility constraints. Often corridors are set in Stage~2 and not expanded further — adapt per your workflow.
  \item The objective uses $\rho_{i,s}=area_i/cost^T_{i,s}$ to prefer cells giving most area per monetary unit, adjusted by suitability.
  \item If you prefer to prioritize particular species in Stage~3, include a multiplicative weight $w_s$ (priority) inside the summand.
\end{itemize}

\section{Constraints}

% -------------------------
% 1. General feasibility & geometry constraints
% -------------------------
\section*{Constraints — General feasibility \& geometry}

\paragraph{Notation reminder (used below).}
We keep the naming conventions introduced earlier:
\[
\begin{aligned}
& I,\; S,\; \mathrm{Adj},\; K_s,\\
& T_{i,s}\in\{0,1\}, \; E_{i,s}\in\{0,1\}, \; C_i\in\{0,1\}, \; use_{i,s}\in\{0,1\},\\
& area_i,\; cost^T_{i,s},\; cost^C_i,\; suit_{i,s}\in[0,1],\\
& n_s,\; m_s,\; \ell_s,\; k_s,\; Rmean_s := n_s\ell_s - m_s,\\
& \rho_{i,s} := \dfrac{area_i}{cost^T_{i,s}} \text{ (precomputed)},\\
& conflict(s,s')\in\{0,1\} \text{ indicates pairwise conflict.}
\end{aligned}
\]

\subsection*{1.1 Variable domains and basic identities}

\begin{align}
& T_{i,s} \in \{0,1\}, \qquad E_{i,s} \in \{0,1\}, \qquad C_i \in \{0,1\}, \qquad use_{i,s} \in \{0,1\},\\[4pt]
& \lambda_{(i,j),s} \in \{0,1\} \quad \forall (i,j)\in\mathrm{Adj},\; s\in S
\qquad\text{(corridor-edge usage binaries, optional)}.
\end{align}

\noindent Basic preservation of original presence:
\begin{align}
& T_{i,s} \;\ge\; orig\_has_{i,s}
\qquad\forall i\in I,\; s\in S. 
\label{eq:preserve_orig}
\end{align}

\noindent Link expanded-only to final membership:
\begin{align}
& E_{i,s} \;\ge\; T_{i,s} - orig\_has_{i,s}
\qquad\forall i,s,
\label{eq:E_lower}\\
& E_{i,s} \;\le\; T_{i,s}, \qquad E_{i,s} \;\le\; 1 - orig\_has_{i,s}
\qquad\forall i,s.
\label{eq:E_upper}
\end{align}
These ensure \(E_{i,s}=1\) exactly when cell \(i\) is newly added for species \(s\).

\subsection*{1.2 Area aggregates}

\noindent Total (final) area by species:
\begin{align}
A_s \;=\; \sum_{i\in I} area_i \; T_{i,s}
\qquad\forall s\in S.
\label{eq:As}
\end{align}

\noindent Total expanded area (for cost accounting) by species:
\begin{align}
ExpArea_s \;=\; \sum_{i\in I} area_i \; E_{i,s}
\qquad\forall s\in S.
\label{eq:ExpArea}
\end{align}

\subsection*{1.3 Budget / cost accounting (global / stage-level)}

\noindent Total adaptation (expansion) cost and corridor cost:
\begin{align}
Cost^{\text{adapt}} &\;=\; \sum_{s\in S}\sum_{i\in I} cost^{T}_{i,s} \; E_{i,s},\\[4pt]
Cost^{\text{corr}} &\;=\; \sum_{i\in I} cost^{C}_{i} \; C_{i}.
\end{align}

\noindent Example global budget constraint (use stage-specific budget where appropriate):
\begin{align}
Cost^{\text{adapt}} + Cost^{\text{corr}} \;\le\; Budget_{\text{total}}.
\label{eq:budget}
\end{align}

\subsection*{1.4 Adjacency / geometry precomputation and usage note}

\noindent We assume the adjacency relation $\mathrm{Adj}$ is precomputed externally (parameter): \((i,j)\in\mathrm{Adj}\) iff cells \(i\) and \(j\) share a side (4-neighborhood).  
Use the set $\mathrm{Adj}$ in later constraints that require adjacency (expansion adjacency, corridor continuity, flow conservation, etc.).

\subsection*{1.5 Corridor logical linking}

\noindent A species may use a corridor cell only if the cell is built as corridor:
\[
use_{i,s} \;\le\; C_i
\qquad \forall i,s.
\]

\noindent
\textbf{Note:} Corridor usage does \emph{not} require territory overlap.  
Connectivity of a species to a corridor is handled solely through the corridor–patch gateway and flow constraints (Section~3).

\subsection*{1.6 Corridor-edge linking (optional)}

\noindent If using edge binaries \(\lambda_{(i,j),s}\) to represent corridor edges used by species \(s\), link them to corridor cells:
\begin{align}
& \lambda_{(i,j),s} \;\le\; C_i,\qquad
  \lambda_{(i,j),s} \;\le\; C_j, \qquad
  \lambda_{(i,j),s} \;\ge\; C_i + C_j - 1,
  \label{eq:edge_C_link}
\end{align}
\noindent and if per-species edge usage must be tied to species accessibility:
\begin{align}
\lambda_{(i,j),s} \;\le\; use_{i,s}, \qquad
\lambda_{(i,j),s} \;\le\; use_{j,s}.
\label{eq:edge_use_link}
\end{align}

\subsection*{1.7 Suitability-based hard exclusions (optional)}

\noindent If you want to {\bf forbid} selection of cells that are unsuitable for a species (hard exclusion), define a binary parameter \(allowed_{i,s}\in\{0,1\}\) by preprocessing:
\[
allowed_{i,s} \;=\; \begin{cases} 1 & \text{if } suit_{i,s} \ge \tau_s,\\
0 & \text{otherwise,} \end{cases}
\]
for some threshold \(\tau_s\). Then impose:
\begin{align}
& T_{i,s} \;\le\; allowed_{i,s}, \qquad E_{i,s} \;\le\; allowed_{i,s}
\qquad\forall i,s.
\label{eq:suit_exclude}
\end{align}

\noindent If you prefer soft exclusion, instead multiply suitability into the objective and do not enforce \eqref{eq:suit_exclude}.

\subsection*{1.8 Conflict / predator–prey overlap protection (do not worsen initial conflicts)}

\noindent Let \(P\) be the set of unordered species-pairs with conflicts:
\[
P \;=\; \{(s,s')\in S\times S : conflict(s,s')=1,\; s<s'\}.
\]
\noindent Precompute the original overlap indicator for each cell and pair:
\[
orig\_overlap_{i,s,s'} \;:=\; orig\_has_{i,s}\cdot orig\_has_{i,s'}
\]
(this is a parameter equal to 1 if both species originally co-occur in cell \(i\), 0 otherwise).

\noindent Introduce linearized overlap binaries for final map:
\[
O_{i,s,s'} \in\{0,1\},\qquad
O_{i,s,s'} \le T_{i,s},\quad O_{i,s,s'} \le T_{i,s'},\quad
O_{i,s,s'} \ge T_{i,s} + T_{i,s'} - 1,
\]
for all \((s,s')\in P,\; i\in I\). These enforce \(O_{i,s,s'} = 1\) iff both species occupy cell \(i\) in the final plan.

\noindent Constraint: do not increase the number of overlapping cells relative to the original map:
\begin{align}
\sum_{i\in I} O_{i,s,s'} \;\le\; \sum_{i\in I} orig\_overlap_{i,s,s'}
\qquad\forall (s,s')\in P.
\label{eq:do_not_worsen_overlap}
\end{align}
This enforces your rule: "If there is any overlap between initial conflicting colonies at the beginning, we ignore it and expand in a way that does not worsen the situation."

\subsection*{1.9 Corridor exclusivity for conflicting species}

\noindent Prevent conflicting species from sharing the {\em same} corridor cell (if required):
\[
use_{i,s} + use_{i,s'} \;\le\; 1 \qquad
\forall (s,s')\in P,\; i\in I.
\label{eq:corridor_exclusive}
\]

\subsection*{1.10 Bookkeeping / useful derived quantities}

\noindent Number of corridor cells usable by species \(s\):
\[
Y_{cells,s} \;=\; \sum_{i\in I} use_{i,s}.
\]

\noindent Number of corridor edges used by species \(s\) (if edges used):
\[
Y_{edges,s} \;=\; \sum_{(i,j)\in\mathrm{Adj}} \lambda_{(i,j),s}.
\]

\subsection*{1.11 Soft enforcement note for tree-like corridors}

\noindent The strict constraint ``no cycles; corridor count \(\le n_s-1\)'' can be computationally heavy.
We therefore recommend implementing this concept as a \emph{soft} preference:
\[
\text{penalize }\; \sum_{(i,j)\in\mathrm{Adj}}\sum_{s\in S}\lambda_{(i,j),s}
\quad\text{in the objective (small weight)}
\]
rather than enforcing a combinatorial cycle-elimination constraint. If you prefer a hard tree constraint later, add connectivity-flow formulations with cut/MTZ-style subtour elimination (but be aware of model size).

\bigskip

% End of general constraints block

% -------------------------
% 2. Territory formation \& expansion constraints
% -------------------------
\section*{Constraints — Territory formation \& expansion}

\paragraph{Notation reminder (new items used here).}
\[
\begin{aligned}
& T_{i,s,k} \in \{0,1\} \text{ for } k\in K_s:
    \text{ cell $i$ is assigned to final territory of original patch } k,\\
& patchOf_{i,s} \text{ : parameter mapping original cell $i$ to its original patch } k\in K_s\\
& \text{ (defined only when } orig\_has_{i,s}=1\text{)},\\
& f_{i,j,s,k} \ge 0 \text{ : continuous flow on directed arc } (i\to j)
    \text{ for patch } k\text{ of species } s,\\
& U^{\text{flow}} \text{ : a large upper bound for flow (suggest } U^{\text{flow}} = |I|\text{).}
\end{aligned}
\]

\subsection*{2.1 Patch assignment and no-merge}

\noindent \textbf{(a) Decompose species membership by original patch.}
\begin{align}
& T_{i,s} \;=\; \sum_{k\in K_s} T_{i,s,k}
    \qquad \forall i\in I,\; s\in S,
\label{eq:Ti_decomp}
\end{align}
with the interpretation that every final membership \(T_{i,s}\) is assigned to exactly one original patch of the same species.

\noindent \textbf{(b) Prevent a cell being assigned to more than one original patch of the same species (no-merge).}
\begin{align}
& \sum_{k\in K_s} T_{i,s,k} \;\le\; 1
    \qquad \forall i\in I,\; s\in S.
\label{eq:no_merge}
\end{align}

\noindent \textbf{(c) Fix original patch assignments for original cells.}
For each original cell \(i\) and species \(s\) let \(k = patchOf_{i,s}\) be its original patch id (parameter). Then:
\begin{align}
& T_{i,s,k} \;=\; orig\_has_{i,s}
    \qquad \forall i\in I,\; s\in S \text{ with } orig\_has_{i,s}=1.
\label{eq:orig_assign}
\end{align}
(If a species does not originally occupy cell \(i\), there is no enforced equality in \eqref{eq:orig_assign}.)

\subsection*{2.2 Per-patch area accounting and A1/A2 split (for Stage 2 reward)}

\noindent \textbf{(a) Patch area variables.}
\[
A_{k,s} \;=\; \sum_{i\in I} area_i \; T_{i,s,k}
        \qquad \forall s\in S,\; k\in K_s.
\label{eq:patch_area}
\]

\noindent \textbf{(b) Per-patch A1/A2 split that implements "reward = 1 up to mean, 0.7 beyond".}
For every original patch \(k\in K_s\) we define:
\begin{align}
& A1_{k,s} + A2_{k,s} \;=\; A_{k,s}, \label{eq:A_split_sum}\\[4pt]
& 0 \;\le\; A1_{k,s} \;\le\; meanArea_s, \qquad A2_{k,s} \;\ge\; 0.
\label{eq:A_split_bounds}
\end{align}
(At optimum, \(A1_{k,s} = \min\{A_{k,s}, meanArea_s\}\) and \(A2_{k,s} = \max\{0, A_{k,s} - meanArea_s\}\).)

\noindent \textbf{(c) Species-level aggregates (useful in objectives).}
\begin{align}
& A_{s} \;=\; \sum_{k\in K_s} A_{k,s}, \qquad
  A1_{s} \;=\; \sum_{k\in K_s} A1_{k,s}, \qquad
  A2_{s} \;=\; \sum_{k\in K_s} A2_{k,s}.
\label{eq:species_agg}
\end{align}

\subsection*{2.3 Minimal / maximum area per patch / species (Stage 1 \& 3)}

\noindent \textbf{(a) Per-species Stage~1 minimal requirement (hard):}
\begin{align}
& A_{s} \;\ge\; n_s \cdot meanArea_s \;-\; m_s \;=\; Rmean_s
    \qquad \forall s\in S,
\label{eq:species_min}
\end{align}
or, if you want to enforce per-patch minima instead, require:
\begin{align}
& A_{k,s} \;\ge\; minArea_{k,s}
    \qquad \forall s\in S,\; k\in K_s.
\label{eq:patch_min}
\end{align}

\noindent \textbf{(b) Per-species maximum cap (Stage~3 constraint):}
\begin{align}
& A_{s} \;\le\; n_s \cdot k_s \;-\; m_s
    \qquad \forall s\in S.
\label{eq:species_max}
\end{align}

\subsection*{2.4 Adjacency-based expansion feasibility (local rule)}

\noindent New expansions must be adjacent to existing territory (original or newly expanded). For each cell \(i\) and species \(s\):
\begin{align}
& E_{i,s} \;\le\; \sum_{j:\,(i,j)\in\mathrm{Adj}} T_{j,s}.
\label{eq:expansion_adj}
\end{align}
This enforces: if \(E_{i,s}=1\) then at least one neighbor \(j\) has \(T_{j,s}=1\) (original or expanded) ensuring local growth only.

\subsection*{2.5 Flow-based connectivity certification for each original patch (guarantees contiguity)}

\noindent Rationale: to ensure that all cells assigned to a particular original patch \(k\in K_s\) form a connected region reachable from the chosen patch root \(r = patchroot_{s,k}\), we use a single-commodity flow construction.

\paragraph{Definitions / bounds.}
\[
\begin{aligned}
& f_{i,j,s,k} \ge 0 \quad \forall (i,j)\in\mathrm{Adj},\; s\in S,\; k\in K_s
\qquad\text{(directed flow variable on edge $i\to j$)},\\
& \text{choose } U^{\text{flow}} = |I| \text{ (or } \sum_i \mathbf{1}\text{)} \text{ as an upper bound.}
\end{aligned}
\]

\paragraph{Edge capacity / activation linking (optional)}  
Optionally, introduce binary \(Z_{(i,j),s,k}\in\{0,1\}\) to indicate edge activation for patch \(k\).
Link flows to activations if desired:
\begin{align}
& f_{i,j,s,k} \;\le\; U^{\text{flow}} \; Z_{(i,j),s,k}, 
\qquad Z_{(i,j),s,k} \;\le\; T_{i,s,k}, \qquad Z_{(i,j),s,k} \;\le\; T_{j,s,k}.
\label{eq:flow_edge_link}
\end{align}
(Using \(Z\)-binaries strengthens the formulation but increases binary count; the formulation remains valid without \(Z\) by using bounds on \(f\) tied to \(T\).)

\paragraph{Flow conservation / demand definition.}
Let \(r = patchroot_{s,k}\). For each node \(i\in I\) and each \(s,k\):
\begin{align}
& \sum_{j:\,(j,i)\in\mathrm{Adj}} f_{j,i,s,k} \;-\; \sum_{j:\,(i,j)\in\mathrm{Adj}} f_{i,j,s,k}
    \;=\; d_{i,s,k},
\label{eq:flow_cons}
\end{align}
where the right-hand side \(d_{i,s,k}\) is defined as:
\[
d_{i,s,k} \;=\;
\begin{cases}
- \displaystyle \sum_{i'\in I\setminus\{r\}} T_{i',s,k} 
    & \text{if } i = r \quad (\text{root supplies total demand}),\\[8pt]
\; T_{i,s,k}      & \text{if } i \ne r \quad (\text{each selected non-root node demands 1 unit}).
\end{cases}
\]
This enforces that for each selected node \(i\ne r\) exactly one unit of flow must reach it from the root \(r\).

\paragraph{Flow-variable bounding by selection (to prevent flow through unselected nodes).}
For every arc \((i,j)\) and patch \(k\):
\begin{align}
& f_{i,j,s,k} \;\le\; U^{\text{flow}} \; T_{i,s,k}, \label{eq:flow_Ti_bound}\\
& f_{i,j,s,k} \;\le\; U^{\text{flow}} \; T_{j,s,k}. \label{eq:flow_Tj_bound}
\end{align}
Thus flows can be nonzero only on arcs whose both endpoints are selected for the same patch (either enforced via these bounds or via the optional \(Z\) variables in \eqref{eq:flow_edge_link}).

\paragraph{Interpretation.}
If a node \(i\) is selected for patch \(k\) then the flow constraints force a path of positive flow from the root \(r\) to \(i\) using only nodes/edges where \(T_{*,s,k}=1\). This certifies contiguity of the final patch \(k\).

\subsection*{2.6 Species absence and preventing expansions for absent species}

\noindent If a species \(s\) has no original patches (\(n_s=0\)), disallow expansion and territory:
\begin{align}
& T_{i,s} \;=\; 0,\qquad E_{i,s} \;=\; 0
    \qquad \forall i\in I,\; \text{if } n_s = 0.
\label{eq:no_species}
\end{align}

\subsection*{2.7 Symmetry-breaking (optional, helps solver)}

\noindent To reduce symmetric solutions among interchangeable patches, you may impose ordering on patch indices using e.g. root-cell ids or patch areas:
\[
A_{k,s} \;\ge\; A_{k+1,s} \qquad \forall s\in S,\; k \text{ ordered by patch id}
\]
(Use only if necessary — adds constraints but can speed up branch-and-bound.)

% End of Territory formation & expansion constraints block

% -------------------------
% 3. Corridor constraints
% -------------------------
% ============================================================
% Corridor Design & Connectivity Constraints
% ============================================================

\section{Corridor Design and Connectivity Constraints}

This section introduces binary corridor–cell variables, species–specific usage variables,
patch–corridor overlap indicators, and the patch–level connectivity system based on the
reachability variables \(Reach_{i,s,k}\). Connectivity is certified deterministically by
propagating reachability through activated corridor edges.

\vspace{6pt}
\subsection{Notation (new variables)}
\[
\begin{aligned}
& C_i \in \{0,1\} && \text{corridor built in cell $i$},\\
& use_{i,s} \in \{0,1\} && \text{species $s$ may use corridor cell $i$},\\
& \lambda_{(i,j),s} \in \{0,1\} && \text{corridor edge $(i,j)\in Adj$ usable by species $s$},\\
& Overlap_{i,s,k} \in \{0,1\} && \text{corridor cell $i$ overlaps patch $k$ of species $s$},\\
& Gateway_{s,k} \in \{0,1\} && \text{patch $k$ of species $s$ has at least one corridor gateway},\\
& Reach_{i,s,k} \in \{0,1\} && \text{cell $i$ is corridor-reachable from patch $k$ of species $s$}.
\end{aligned}
\]

% ------------------------------------------------------------
\subsection{Corridor Cell Usage}

A species may use a corridor cell only if the corridor cell exists:
\begin{equation*}
use_{i,s} \;\le\; C_i \qquad \forall i \in I,\; s \in S.
\end{equation*}

Optionally, require that species may use a corridor cell only if it overlaps their territory
(unless explicitly allowed):
\begin{equation*}
use_{i,s} \;\le\; T_{i,s} + Allow_{i,s}, \qquad Allow_{i,s} \in \{0,1\}.
\end{equation*}

% ------------------------------------------------------------
\subsection{Corridor Edges}

Corridor edges may be used only if both endpoint corridor cells are built:
\begin{align}
& \lambda_{(i,j),s} \;\le\; C_i, \qquad
  \lambda_{(i,j),s} \;\le\; C_j, \\
& \lambda_{(i,j),s} \;\ge\; C_i + C_j - 1,
  \qquad \forall (i,j)\in Adj,\; s\in S.
\end{align}

Species may use an edge only if they may use both endpoint cells:
\begin{align}
\lambda_{(i,j),s} \;\le\; use_{i,s}, \qquad
\lambda_{(i,j),s} \;\le\; use_{j,s}
\qquad \forall (i,j)\in Adj,\; s\in S.
\end{align}

% ------------------------------------------------------------
\subsection{Corridor--Patch Overlap and Gateways}

Corridor cell $i$ overlaps patch $k$ of species $s$ if and only if both conditions hold:
\begin{align}
& Overlap_{i,s,k} \;\le\; C_i, \\
& Overlap_{i,s,k} \;\le\; T_{i,s,k}, \\
& Overlap_{i,s,k} \;\ge\; C_i + T_{i,s,k} - 1,
\qquad \forall i\in I,\; s\in S,\; k\in K_s.
\end{align}

A patch has a corridor gateway if at least one corridor cell overlaps it:
\begin{align}
& \sum_{i \in I} Overlap_{i,s,k} \;\ge\; Gateway_{s,k}, \\
& \sum_{i \in I} Overlap_{i,s,k} \;\le\; U^{gw} \, Gateway_{s,k},
\end{align}
where $U^{gw}$ is an upper bound on possible gateway cells (e.g.\ patch area).

% ------------------------------------------------------------
\subsection{Corridor Reachability and Patch Connectivity}

\paragraph{Initial reachability from a patch.}
Any gateway cell of patch $k$ is initially reachable from patch $k$:
\begin{equation*}
Reach_{i,s,k} \;\ge\; Overlap_{i,s,k} \qquad \forall i,s,k.
\end{equation*}

\paragraph{Propagation of reachability through corridor edges.}
Reachability spreads from $i$ to $j$ only if the corridor edge $(i,j)$
is active and usable by species $s$:
\begin{align}
Reach_{j,s,k} &\;\ge\; Reach_{i,s,k} + \lambda_{(i,j),s} - 1,\\
Reach_{i,s,k} &\;\ge\; Reach_{j,s,k} + \lambda_{(i,j),s} - 1,
\qquad \forall (i,j)\in Adj,\; s\in S,\; k\in K_s.
\end{align}

\paragraph{Patch-to-root connectivity certification.}
Let $k_r$ denote the chosen root patch of species $s$.
Patch $k$ is considered corridor-connected to the root if a corridor cell that overlaps
patch $k$ is reachable from any corridor cell overlapping the root:
\begin{equation}
conn_{s,k} \;\le\;
\sum_{i\in I} Reach_{i,s,k_r} \cdot Overlap_{i,s,k},
\qquad \forall s\in S,\; k\in K_s.
\end{equation}

Additionally, a patch cannot be connected unless it has a gateway:
\begin{equation}
conn_{s,k} \;\le\; Gateway_{s,k} \qquad \forall s,k.
\end{equation}

% ------------------------------------------------------------
\subsection{Corridor Exclusivity for Predator--Prey Species}

For each predator–prey pair $(s,s')\in P$, disallow simultaneous use of a corridor cell:
\begin{equation}
use_{i,s} + use_{i,s'} \;\le\; 1
\qquad
\forall i\in I,\; (s,s')\in P.
\end{equation}

% ------------------------------------------------------------
\subsection{Corridor Construction Cost}

Corridor construction cost is accumulated as:
\begin{equation}
Cost^{corr} \;=\;
\sum_{i\in I} cost_i^C \, C_i.
\end{equation}

% ------------------------------------------------------------
\subsection{Soft Penalty for Excessive Corridor Edges}

To encourage sparse, tree-like corridor networks while still allowing cycles:
\begin{equation}
Penalty_{\mathrm{edges}}
\;=\;
\gamma_{\mathrm{edge}}
\sum_{s\in S} \sum_{(i,j)\in Adj} \lambda_{(i,j),s}.
\end{equation}

\subsection{Gateways have distance 0 (if Overlap exists).}

This forces any corridor cell that overlaps a patch k to be a root (distance 0) for propagation.
\begin{equation*}
dist_{i,s,k} \;\le\; L_{\max}\,(1 - Overlap_{i,s,k}) \;+\; 0
\qquad\forall i,s,k. 
\end{equation*}
Explanation: when $Overlap_{i,s,k}=1$, $dist_{i,s,k} \le 0 \Rightarrow dist=0$.

\subsection{Upper bound for all dist variables (redundant but explicit)}

\begin{equation*}
dist_{i,s,k} \;\le\; L_{\max}
\qquad\forall i,s,k.
\end{equation*}

\subsection{Propagation constraint across an active corridor edge:}

if edge $(i,j)$ is active for species $s$, then $dist_j <= dist_i + 1$.
Use $Big-M = L_{max}$ to deactivate when edge not active.
\begin{equation*}
dist_{j,s,k} \;\le\; dist_{i,s,k} \;+\; 1 \;+\; L_{\max}\,(1 - \lambda_{(i,j),s})
\qquad\forall (i,j)\in Adj,\; s\in S,\; k\in K_s.  
\end{equation*}

\subsection{Symmetric direction (ensure propagation both ways if needed):}

\begin{equation*}
dist_{i,s,k} \;\le\; dist_{j,s,k} \;+\; 1 \;+\; L_{\max} \cdot (1 - \lambda_{(i,j),s})
\qquad\forall (i,j)\in Adj,\; s\in S,\; k\in K_s.
\end{equation*}

\subsection{Link reachability to distance:} 

if cell \(i\) is reachable (\(Reach_{i,s,k}=1\)) from patch \(k\),
its distance must be \(\le L_{\max}\). (The \texttt{dist} variable is already bounded by \(\,L_{\max}\,\), but this constraint ties Reach to dist.)
\begin{equation*}
dist_{i,s,k} \;\le\; L_{\max} \;-\; (1 - Reach_{i,s,k}) \cdot L_{\max}
\qquad\forall i,s,k.
\end{equation*}

This constraint is equivalent to: $Reach_{i,s,k}=1 => dist_{i,s,k} <= L_{max}$.
(When $Reach=0$ the $RHS = 0$, so it does not enforce anything active beyond nonnegativity;
the dist variable's domain still applies.)

\subsection{(Optional) force Reach to be consistent with finite distances:}

if $dist <= L_{max}$ then Reach may be 1; otherwise Reach must be 0.
We enforce the implication: $dist_{i,s,k} <= L_max * Reach_{i,s,k}$.
\begin{equation*}
dist_{i,s,k} \;\le\; L_{\max} \; Reach_{i,s,k}
\qquad\forall i,s,k.
\end{equation*}

% End of Corridor Section

% -------------------------
% 4. Predator–prey / interaction constraints
% -------------------------
\section*{Constraints — Predator–prey and species interaction}

\paragraph{Interpretation and scope.}
We assume a set of predator–prey pairs \(P_{\text{pred}} \subseteq S\times S\) where the first element is the predator (e.g. \(M\)) and the second is the prey (e.g. \(E\) or \(O\)). For all \((p,r)\in P_{\text{pred}}\) the model must ensure that final planning does not \emph{worsen} predator–prey co-occurrence at the cell level: no new cell-level overlaps between predator \(p\) and prey \(r\) may be created (though existing overlaps that are present in the original map are tolerated).

\subsection*{4.1 Cell-level no-new-overlap for predator–prey pairs (strict)}

\noindent Precompute original co-occurrence parameter:
\[
orig\_overlap_{i,p,r} \;=\; orig\_has_{i,p}\cdot orig\_has_{i,r},
\qquad \forall i\in I,\; (p,r)\in P_{\text{pred}}.
\]

\noindent Linearized final overlap indicator (repeated for clarity):
\[
\begin{aligned}
& O_{i,p,r} \in\{0,1\},\\[4pt]
& O_{i,p,r} \;\le\; T_{i,p},\qquad O_{i,p,r} \;\le\; T_{i,r},\\[4pt]
& O_{i,p,r} \;\ge\; T_{i,p} + T_{i,r} - 1.
\end{aligned}
\]

\noindent \textbf{Strict no-new-overlap constraint (per cell):}
\begin{align}
O_{i,p,r} \;\le\; orig\_overlap_{i,p,r}
\qquad \forall i\in I,\; (p,r)\in P_{\text{pred}}.
\label{eq:pred_cell_no_new}
\end{align}

\noindent This enforces that a predator \(p\) and prey \(r\) can co-occur in cell \(i\) in the final plan only if they already co-occurred there originally.

\subsection*{4.2 No expansion of predator into prey-occupied cells (alternative / redundant)}

\noindent An equivalent linear form (redundant if \eqref{eq:pred_cell_no_new} is used) that explicitly prevents the predator from occupying any cell that the prey occupies unless originally co-occupied:
\[
T_{i,p} \;\le\; 1 - (T_{i,r} - orig\_has_{i,p}\cdot orig\_has_{i,r})
\qquad \forall i,\; (p,r)\in P_{\text{pred}}.
\]
This is stronger-looking in algebraic form but logically equivalent to the cell-level overlap prohibition above.

\subsection*{4.3 Corridor exclusivity reiterated}

\noindent Predator and prey should not be permitted to use the same corridor cells:
\[
use_{i,p} + use_{i,r} \;\le\; 1 \qquad \forall i\in I,\; (p,r)\in P_{\text{pred}}.
\]
This was already included in general constraints but is repeated here for emphasis.

\subsection*{4.4 Prevent predator expansions that simultaneously overlap original and newly-expanded prey territory (optional special rule)}

\noindent You requested a special rule that prevents predator \(p\) from expanding into cells that would cause it to intersect both original and newly-expanded territories of prey \(r\). One linear implementation is:

\[
\begin{aligned}
& \text{For each cell } i:\quad
E_{i,p} + orig\_has_{i,r} + E_{i,r} \;\le\; 2.
\end{aligned}
\]

\noindent Explanation: If both \(orig\_has_{i,r}=1\) (prey originally present) and \(E_{i,r}=1\) (prey newly expanded into the cell), then \(orig\_has_{i,r} + E_{i,r} \ge 1\); the inequality above prevents \(E_{i,p}=1\) when both are present in a way that would make predator expansion coincide with prey original+expanded presence.  
This rule is optional and should be used only if the biological justification is necessary — otherwise the strict per-cell no-new-overlap (Section~4.1) is sufficient.

\subsection*{4.5 Optional buffer-zone (neighborhood) constraints for prey protection}

\noindent If you wish to enforce a one-cell buffer (no predator expansion into any cell adjacent to a prey's final territory), include this neighbor-based linear constraint for all \((p,r)\in P_{\text{pred}}\):
\[
E_{i,p} \;\le\; 1 - \max_{j:\,(i,j)\in\mathrm{Adj}} T_{j,r},
\qquad \forall i\in I.
\]
\noindent Linearized as:
\[
E_{i,p} \;\le\; 1 - B_{i,r}, \qquad
B_{i,r} \;\ge\; T_{j,r} \quad \forall j:(i,j)\in\mathrm{Adj},\quad
B_{i,r}\in\{0,1\}.
\]
This forces predator \(p\) to avoid any cell adjacent to the prey's final territory. Use only if buffer protection is required.

\subsection*{4.6 Remarks on alternatives and model behavior}

\begin{itemize}
  \item Using per-cell no-new-overlap \eqref{eq:pred_cell_no_new} is conservative and safe: predators never newly co-occur with prey in any cell. If you prefer a more flexible policy (e.g. allow some controlled new overlaps as long as total overlap does not increase), replace the per-cell constraint by the earlier sum-level constraint:
  \[
  \sum_{i\in I} O_{i,p,r} \;\le\; \sum_{i\in I} orig\_overlap_{i,p,r}.
  \]
  \item Corridor exclusivity is required to prevent predator and prey from using the same corridor cells and potentially transmitting predation through the same route. This is enforced by the `use` variables constraint above.
  \item If some initial overlaps exist (orig\_overlap = 1), the model will not remove them (we preserve originals) but will not create any new overlapping cells involving predator–prey pairs.
\end{itemize}


\end{document}
